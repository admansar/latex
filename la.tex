    \documentclass{beamer}
\usetheme{Madrid} % Theme

\title{DÉRIVATION DES DISTRIBUTIONS et Opérations élémentaires}
\author{}
\date{\today}

\begin{document}

\frame{\titlepage} % Title slide

\begin{frame}
\frametitle{Agenda}
\tableofcontents % Table of contents slide
\end{frame}

% \section{Introduction}
% \begin{frame}
% \frametitle{Introduction}
% Dans ce chapitre on définit la dérivée au sens des distributions.
% %kml hnaa introo
% \end{frame}



% \section{Content}
\section{Dérivation des distributions}
\begin{frame}
\frametitle{Dérivation des distributions}
\begin{block}{definition}
Soit $f$ une fonction de classe $C^1$, On appelle dérivée $T'$ d'une distribution $T$ :\\
\hspace{1cm}\hspace{1cm}\hspace{1cm}\hspace{1cm} $\forall \varphi \in \mathcal{D}, \langle T', \varphi \rangle = -\langle T, \varphi' \rangle$
\end{block}
\\
\textbf{Proposition :}
\\
Toute distribution admet des dérivées de tout ordre qui sont aussi des distributions.
\\
\begin{block}{Définition :} \\
On dit qu'une distribution $S$ est une primitive d'une distribution $T$ si et seulement si $T' = S$.
\end{block}
% \begin{alertblock}{Alert Block}
% This is an alert block.
% \end{alertblock}

% \begin{exampleblock}{Example Block}
% This is an example block.
% \end{exampleblock}

\end{frame}

\begin{frame}

\textbf{Exemples : Dérivée de la fonction d'Heaviside}

\vspace{0.5cm} % Adjust the space as needed

La fonction d'Heaviside (dite échelon unité) est définie par :

\vspace{0.3cm} % Adjust the space as needed

\hspace{2cm} $H(x) = \begin{cases}
\hspace{2cm} 0 & \text{si } x < 0 \\
\hspace{2cm} 1 & \text{si } x > 0
\end{cases}$
\vspace{0.5cm} % Adjust the space as needed


\\
Au sens des fonctions, la dérivée de $H(x)$ n'existe pas au point $x = 0$. Mais au sens des distributions, on a pour $\varphi \in \mathcal{D}$

\vspace{0.3cm} 
\\

\langle H', \varphi \rangle &= - \langle H, \varphi' \rangle \\
\vspace{0.3cm} 
\hspace{1cm} = -\int_{-\infty}^{+\infty} H(x) \varphi'(x) \, dx
\\
\vspace{0.3cm} 
\hspace{1cm} = -\int_{-\infty}^{0} H(x) \varphi'(x) \, dx - \int_{0}^{+\infty} H(x) \varphi'(x) \, dx
\\ 
\vspace{0.3cm} 
\hspace{1cm} = -\int_{0}^{+\infty} \varphi'(x) \, dx = \varphi'(0) = \langle \delta, \varphi \rangle
\vspace{0.5cm} 

\text{d'où } H' = \delta

\\
\vspace{0.3cm} 
\end{frame}
\begin{frame}

\textbf{Extension au cas de plusieurs variables }
\\
\text{Dans le cas de plusieurs variables, on définit la dérivée } \dfrac{\partial T}{\partial x_i}\\\text{ d'une distribution } T, par :
\vspace{0.3cm} 
\\
\hspace{2cm} \langle \dfrac{\partial T}{\partial x_i}, \varphi \rangle = -\langle T, \dfrac{\partial \varphi}{\partial x_i} \rangle , \hspace {0.7cm} i = 1, 2, ..., m.
\\
Plus généralement, on a :
\\ 
\vspace{0.3cm} 
\hspace{2cm} \langle D^k, \varphi \rangle = (-1)^{\left| k \right|} \langle T, D^k\varphi \rangle
\\
\vspace{0.3cm} 
où k = (k_1, k_2, ..., k_m) \in \mathbb{N}^m  avec \left| k \right| = k_1+k_2+...k_m et 
\\
\vspace{0.3cm} 
\hspace{2cm}D^k = \dfrac{\partial^k_1}{\partial {x_1}^{k_1}}.\dfrac{\partial^k_2}{\partial {x_2}^{k_2}}...\dfrac{\partial^k_m}{\partial {x_m}^{k_m}} = 
\dfrac{\partial^{k_1 + k_2 + ... + k_m}}{\partial {x_1}^{k_1}\partial {x_2}^{k_2}...\partial {x_m}^{k_m}}

\end{frame}

\begin{frame}
\textbf{Exemples :}
    Soient \( H(x) \) la fonction d'Heaviside, \( \alpha \)  une constante réelle. Prouver, au sens des distributions, la relation suivante :
\\
\vspace{0.3cm} 
  \hspace{2cm}\left(\frac{d}{dx} - \alpha\right) H(x) e^{\alpha x} = \delta
\vspace{0.3cm} Ω
\\
  \textbf{Solution :}
L'énoncé initial est :

\[ \left(\frac{d}{dx} - \alpha\right) H(x) e^{\alpha x} = \delta(x) \]

Nous allons utiliser l'intégration par parties pour prouver cette égalité.

\\
\begin{align*}
&\int_{-\infty}^{\infty} \left(\frac{d}{dx} - \alpha\right) H(x) e^{\alpha x} \phi(x) \,dx = \int_{-\infty}^{\infty} \delta(x) \phi(x) \,dx = \phi(0)
\end{align*}
\\
\end{frame}
\begin{frame}

Commençons par calculer le côté gauche :

\\
\begin{align*}
&\int_{-\infty}^{\infty} \left(\frac{d}{dx} - \alpha\right) H(x) e^{\alpha x} \phi(x) \,dx \\
&= \int_{-\infty}^{\infty} \frac{d}{dx}\left(H(x) e^{\alpha x}\right) \phi(x) \,dx - \alpha \int_{-\infty}^{\infty} H(x) e^{\alpha x} \phi(x) \,dx
\end{align*}
\\


En utilisant \(H'(x) = \delta(x)\) :
% \begin{frame}
% \end{frame}
\\

\begin{align*}
&\int_{-\infty}^{\infty} \frac{d}{dx}\left(H(x) e^{\alpha x}\right) \phi(x) \,dx \\
&= \int_{-\infty}^{\infty} \delta(x) e^{\alpha x} \phi(x) \,dx + \alpha \int_{-\infty}^{\infty} H(x) e^{\alpha x} \phi(x) \,dx
\end{align*}```

En simplifiant :
\end{frame}
\begin{frame}
\begin{align*}
&\alpha \int_{-\infty}^{\infty} H(x) e^{\alpha x} \phi(x) \,dx = \phi(0) \\
&\int_{-\infty}^{\infty} H(x) e^{\alpha x} \phi(x) \,dx = \frac{1}{\alpha}\phi(0) \\
&\int_{-\infty}^{\infty} \delta(x) \phi(x) \,dx = \phi(0)
\end{align*}
\\

Par conséquent, nous avons montré que \( \left(\frac{d}{dx} - \alpha\right) H(x) e^{\alpha x} = \delta(x) \) en termes de distributions en utilisant l'intégration par parties.

\\
ou d'une autre facon :
\\
\\
\vspace{0.3cm} 
  \hspace{2cm}\left(\frac{d}{dx} - \alpha\right) H(x) e^{\alpha x} = \frac{d}{dx} \left(H(x) e^{\alpha x}\right) - \alpha H(x) e^{\alpha x}

\\
\vspace{0.3cm} 
% \begin{align*}

 En utilisant cette expression, reprenons l'équation initiale : 
 \\
 \vspace{0.3cm}
 \hspace{2cm} \dfrac{d}{dx} \left(H(x) e^{\alpha x}\right) = \delta(x) e^{\alpha x} + H(x) \alpha e^{\alpha x}
\\
\vspace{0.3cm}

% \end{align*}

\end{frame}

\begin{frame}
Ainsi, après réarrangement, on obtient effectivement :

\vspace{0.3cm} 
\hspace{2cm} $\left(\frac{d}{dx} - \alpha\right) H(x) e^{\alpha x} = \delta(x) e^{\alpha x} + \alpha H(x) e^{\alpha x} - \alpha H(x) e^{\alpha x}$
\vspace{0.3cm} 
\hspace{2cm} $\left(\frac{d}{dx} - \alpha\right) H(x) e^{\alpha x} = \delta(x) e^{\alpha x}$
\vspace{0.3cm}

La présence de $e^{\alpha x}$ ne change pas la nature de la distribution de Dirac, donc on peut simplifier $\delta(x) e^{\alpha x}$ en $\delta(x)$.
\\
\\\
donc, la relation au sens des distributions est bien établie.
\end{frame}




\section{Opérations élémentaires}
\begin{frame}
\frametitle{Opérations élémentaires}
% \frametitle{Lists}
% \begin{itemize}
%     \item Item 1
%     \item Item 2
%     \item Item 3
% \end{itemize}

% \begin{enumerate}
%     \item First
%     \item Second
%     \item Third
% \end{enumerate}
\begin{block}{definition}
Le produit d'une distribution quelconque \( T \) par une fonction \( g \) de classe \( C^{\infty} \) est défini par :

\[
\langle T \cdot g, \varphi \rangle = \langle T, g \cdot \varphi \rangle
\]

où \( \langle T, \varphi \rangle \) est l'action de la distribution \( T \) sur la fonction test \( \varphi \), et \( g \cdot \varphi \) est le produit de la fonction \( g \) par la fonction test \( \varphi \).

\end{block}
\\
\textbf{Exemple :}
Considérons la distribution de Dirac \( \delta(x) \) et une fonction \( g(x) \) quelconque, par exemple \( g(x) = x^2 \), qui est de classe \( C^{\infty} \) sur son domaine.

La définition du produit de \( \delta(x) \) par \( g(x) \) est donnée par :

\[
\langle \delta(x) \cdot g(x), \varphi(x) \rangle = \langle \delta(x), g(x) \cdot \varphi(x) \rangle
\]
\end{frame}
\begin{frame}
Pour une fonction test \( \varphi(x) \), cette équation se traduit par :

\[
\langle \delta(x) \cdot x^2, \varphi(x) \rangle = \langle \delta(x), x^2 \cdot \varphi(x) \rangle
\]

En utilisant les propriétés de la distribution de Dirac, on sait que \( \langle \delta(x), x^2 \cdot \varphi(x) \rangle = (g.\varphi)(0) = 0 \) pour toute fonction test \( \varphi(x) \), car \( x^2 \cdot \varphi(x) \) s'annule en \( x = 0 \).

Ainsi, le produit de la distribution de Dirac \( \delta(x) \) par la fonction \( g(x) = x^2 \) est égal à la distribution nulle. Cet exemple montre comment le produit d'une distribution par une fonction de classe \( C^{\infty} \) est défini et comment il peut être évalué dans certains cas.
\end{frame}
\begin{frame}
\begin{block}{proposition :}
Les solutions de \( xT = 0 \) dans \( \mathcal{D}' \) sont les distributions \( T = c\delta \) où \( c \in \mathbb{C} \).
\textbf{en générale}, La solution de l'équation \( xT = S \), où \( S \) est une distribution donnée, est égale à la somme d'une solution particulière \( T_0 \) de cette équation et de la solution générale de l'équation homogène. On en déduit de la proposition précédente que cette solution s'écrit sous la forme : \( T = c \delta + T_0 \), où \( c \in \mathbb{C} \). En effet, si \( T \) et \( T_0 \) sont des solutions de \( xT = S \), alors \( x(T-T_0) = 0 \), d'où \( T-T_0 = c\delta \), \( c \in \mathbb{C} \).

\end{block}

\end{frame}
\begin{frame}
\begin{block}{définition :}

Soit \( f(x) \) une fonction localement sommable et \( f \) la distribution qui lui est associée. La distribution associée à \( f(x - a) \), où \( a \) est une constante, est définie comme la translation de la distribution \( f \) par \( a \). On la note souvent \( T_a(f) \).

La translation d'une distribution \( f \) par \( a \) agit sur une fonction test \( \varphi(x) \) comme suit :

\[ \langle T_a(f), \varphi(x) \rangle = \langle f(x - a), \varphi(x) \rangle = \langle f(x), \varphi(x + a) \rangle \]

\end{block}

\textbf{Exemple :}
    Considérons la distribution de Dirac \( \delta(x) \) et regardons comment la distribution \( \delta(x - a) \) agit sur une fonction test \( \varphi(x) \). La distribution \( \delta(x - a) \) représente une translation de la distribution de Dirac de \( a \) unités vers la droite.

La façon dont cette distribution agit sur une fonction test \( \varphi(x) \) est définie par :

\[ \langle \delta(x - a), \varphi(x) \rangle = \varphi(a) \]

\end{frame}
\begin{frame}

\begin{definition}
Une distribution \( T \) est dite périodique de période \( a \) si elle satisfait la propriété suivante pour toute fonction test \( \varphi(x) \) :

\[ \langle T(x), \varphi(x) \rangle = \langle T(x), \varphi(x + a) \rangle  = \langle T(x - a), \varphi(x) \rangle\]
ou encore :
\[ \langle T(x), \varphi(x + a) - \varphi(x) \rangle = 0 \]

\end{definition}

\end{frame}
\begin{frame}


\begin{block}{Définition du Changement d'échelle :}

\text{Soit } T \text{ une distribution. Le changement d'échelle est défini comme suit :}

\vspace{0.2cm}\hspace{2.5cm} \forall a \neq 0, \quad \langle T, \varphi(ax) \rangle = \frac{1}{|a|} \langle T, \varphi\left(\frac{x}{a}\right) \rangle
\end{block}
\\\\\\
\end{frame}


\begin{frame}
\begin{definition}
La transposée \( \check{T} \) d'une distribution \( T \) est définie par la relation suivante pour toute fonction test \( \varphi \) :

\[ \langle \check{T}, \varphi \rangle = \langle T, \check{\varphi} \rangle  \hspace{1cm} \forall \varphi \in \mathcal{D} \]

Une distribution T est dite paire si \( \check{T}  = T \), c'est-à-dire :
\[ \langle \check{T}, \varphi \rangle = \langle T, \varphi \rangle  \hspace{1cm} \forall \varphi \in \mathcal{D} \]
et elle est dite impaire si \( \check{T}  = -T \), c'est-à-dire  :
\[ \langle \check{T}, \varphi \rangle = -\langle T, \varphi \rangle  \hspace{1cm} \forall \varphi \in \mathcal{D} \]

\end{definition}


\end{frame}

% \section{Conclusion}
% \begin{frame}
% \frametitle{Conclusion}
% Thank you for your attention!
\begin{frame}
\[\textbf{merci pour votre attention !}\]
\end{frame}   

\end{document}
