    \documentclass{beamer}
\usetheme{Madrid} % Theme

\title{DÉRIVATION DES DISTRIBUTIONS et Opérations élémentaires}
\author{}
\date{\today}

\begin{document}

\frame{\titlepage} % Title slide

\begin{frame}
\frametitle{Agenda}
\tableofcontents % Table of contents slide
\end{frame}

\section{Introduction}
\begin{frame}
\frametitle{Introduction}
Dans ce chapitre on définit la dérivée au sens des distributions.
%kml hnaa introo
\end{frame}



\section{Content}
\subsection{Dérivation des distributions}
\begin{frame}
\frametitle{Blocks}
\begin{block}{definition}
Soit $f$ une fonction de classe $C^1$, On appelle dérivée $T'$ d'une distribution $T$ :\\
\hspace{1cm}\hspace{1cm}\hspace{1cm}\hspace{1cm} $\forall \varphi \in \mathcal{D}, \langle T', \varphi \rangle = -\langle T, \varphi' \rangle$
\end{block}
\\
\textbf{Proposition :}
\\
Toute distribution admet des dérivées de tout ordre qui sont aussi des distributions.
\\
\begin{block}{Définition :} \\
On dit qu'une distribution $S$ est une primitive d'une distribution $T$ si et seulement si $T' = S$.
\end{block}
% \begin{alertblock}{Alert Block}
% This is an alert block.
% \end{alertblock}

% \begin{exampleblock}{Example Block}
% This is an example block.
% \end{exampleblock}

\end{frame}

\begin{frame}

\textbf{Exemples : Dérivée de la fonction d'Heaviside}

\vspace{0.5cm} % Adjust the space as needed

La fonction d'Heaviside (dite échelon unité) est définie par :

\vspace{0.3cm} % Adjust the space as needed

\hspace{2cm} $H(x) = \begin{cases}
\hspace{2cm} 0 & \text{si } x < 0 \\
\hspace{2cm} 1 & \text{si } x > 0
\end{cases}$
\vspace{0.5cm} % Adjust the space as needed


\\
Au sens des fonctions, la dérivée de $H(x)$ n'existe pas au point $x = 0$. Mais au sens des distributions, on a pour $\varphi \in \mathcal{D}$

\vspace{0.3cm} 
\\

\langle H', \varphi \rangle &= - \langle H, \varphi' \rangle \\
\vspace{0.3cm} 
\hspace{1cm} = -\int_{-\infty}^{+\infty} H(x) \varphi'(x) \, dx
\\
\vspace{0.3cm} 
\hspace{1cm} = -\int_{-\infty}^{0} H(x) \varphi'(x) \, dx - \int_{0}^{+\infty} H(x) \varphi'(x) \, dx
\\ 
\vspace{0.3cm} 
\hspace{1cm} = -\int_{0}^{+\infty} \varphi'(x) \, dx = \varphi'(0) = \langle \delta, \varphi \rangle
\vspace{0.5cm} 

\text{d'où } H' = \delta

\\
\vspace{0.3cm} 
\end{frame}
\begin{frame}

\textbf{Extension au cas de plusieurs variables }
\\
\text{Dans le cas de plusieurs variables, on définit la dérivée } \dfrac{\partial T}{\partial x_i}\\\text{ d'une distribution } T, par :
\vspace{0.3cm} 
\\
\hspace{2cm} \langle \dfrac{\partial T}{\partial x_i}, \varphi \rangle = -\langle T, \dfrac{\partial \varphi}{\partial x_i} \rangle , \hspace {0.7cm} i = 1, 2, ..., m.
\\
Plus généralement, on a :
\\ 
\vspace{0.3cm} 
\hspace{2cm} \langle D^k, \varphi \rangle = (-1)^{\left| k \right|} \langle T, D^k\varphi \rangle
 

\end{frame}




\subsection{Opérations élémentaires}
\begin{frame}
\frametitle{Lists}
\begin{itemize}
    \item Item 1
    \item Item 2
    \item Item 3
\end{itemize}

\begin{enumerate}
    \item First
    \item Second
    \item Third
\end{enumerate}
\end{frame}

\section{Conclusion}
\begin{frame}
\frametitle{Conclusion}
Thank you for your attention!
\end{frame}

\end{document}
